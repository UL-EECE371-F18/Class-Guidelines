\documentclass{article}
\usepackage[utf8]{inputenc}

\title{Project Outline \& Guidelines}
\author{Omar Eldash }
\date{September 2018}

\usepackage{natbib}
\usepackage{graphicx}

\begin{document}

\maketitle

\section{Teams will be labeled as}

Team A \\
Team B \\
Team C (lone ranger) \\

\section{Guidelines}
The project shall consist of the implementation and adoption of:\\
-	Communication protocols (1 Protocol is enough for Team C)\\
-	Computation intensive process (less intensity for Team C)\\
-	ML Algorithm implemented and coordinated between UC and FPGA (Team C can implement a simpler one on a single platform)\\
-	Multiple sensors and actuators to be used (Team C can choose to work with either sensing or actuation\\

\section{Reporting}

At each stage of the project, It will be good to report part on its own before final report \\
-	I expect monthly reports of utilization, timing, power consumption, challenges and remaining tasks. Reports are expected on such dates \\
	-\hspace{1cm}  October 2nd\\
    -\hspace{1cm}Nov. 6th \\
	-\hspace{1cm}Final report at the end of the semester\\
-	It’s recommended to use LaTex for reporting, which would be an additional experience. It’s not mandatory. To facilitate this I will send a template that should make things much easier.\\

\section{Project Ideas}

Identifying objects using camera as part of a bigger system \\
	Systems may include robotics/drones/health/etc. \\
Identification of different types of items and optimizing/replacing FPGA implementation based on each of those. \\
	Example for that is having a high and low quality versions of identification algorithm\\
 	Another example is to identify two objects of different nature that requires different algorithms (like voice/photo , heart pulse/photo tracking)



\end{document}
